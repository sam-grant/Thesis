\begin{abstract}

  The Fermilab Muon $g-2$ experiment aims to measure the anomalous magnetic moment of the muon, $a_{\mu}$, to a precision of 140 parts-per-billion (ppb), and conduct a world-leading search for the electric dipole moment (EDM) of the muon, $d_{\mu}$. At the time of writing, the combination of the first result for $a_{\mu}$ from Fermilab, published in 2021 \cite{SummaryRun1}, and the final result from the Muon $g-2$ experiment at Brookhaven National Laboratory (BNL) \cite{BNLFinalReport}, presents a $4.2\sigma$ tension with the Standard Model (SM) prediction \cite{aMuSM}. If this discrepancy persists, it will require a resolution in the form of the discovery of new physics beyond the SM (BSM). Alongside this, any observation of a muon EDM would constitute a discovery of new physics outright, and would provide a new source of charge-parity (CP) violation from outside the SM. Fermilab aims to improve upon the current muon EDM upper limit of $|d_{\mu}|<1.8\times10^{-19} \,e\cdot\text{cm (95\% C.L.)}$, set by BNL \cite{BNLEDM}, by at least an order of magnitude. 

  In this thesis, a blinded search for $d_{\mu}$ in E989 Run-1 is presented. In support of this, the development and execution of a novel technique for measuring the radial component of the storage ring magnetic field to a precision of $\leq1$ parts-per-million (ppm) is described. The work ensures that the radial magnetic field, which can mimic a signal from a non-vanishing muon EDM, does not present a limiting source of systematic uncertainty in the overall search for $d_{\mu}$ at E989. The production and analysis of large-scale Monte Carlo simulation is also outlined, which is essential for the characterisation of sources of systematic uncertainty in the EDM search, and, critically, the observed dilution of the EDM signal. Finally, blinded results from the search for $d_{\mu}$ using data from straw tracker detectors in the four datasets comprising E989 Run-1 are presented, with an estimated total uncertainty of $\pm1.04\times10^{-19} \,e\cdot\text{cm}$. This uncertainty is used to derive the expected upper limit on the muon EDM of $|d_{\mu}| <  2.0\times10^{-19} \,e\cdot\text{cm (95\% C.L.)}$, which improves upon the upper limit from the equivalent traceback detector analysis performed at the BNL Muon $g-2$ experiment \cite{BNLEDM}. 

\end{abstract}