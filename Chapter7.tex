\chapter{Conclusion}\label{chap:7}

The Fermilab Muon $g-2$ experiment (E989) aims to measure the anomalous magnetic moment of the muon, $a_{\mu}$, to a precision of 140 parts-per-billion (ppb), and conduct a world-leading search for the muon electric dipole moment (EDM), $d_{\mu}$, where any observation of a non-vanishing muon EDM would indicate a source of CP violation from new physics beyond the Standard Model. This thesis is focused on advancing the latter goal, presenting a blinded search for $d_{\mu}$ in E989 Run-1, which is supported by studies with large-scale Monte Carlo datasets, along with a new determination of one of the key systematic effects which will impact the ultimate sensitivity to $d_{\mu}$ at E989.

In the E989 storage ring, a muon EDM would manifest as a tilt in the muon spin precession plane, resulting in an oscillation in the average vertical angle of decay positrons which may be measured directly by use of straw tracker detectors. A potentially limiting source of systematic uncertainty in this measurement is the presence of a non-zero radial component of the main magnetic field, which, by tilting the spin precession plane in the same manner as a muon EDM, can result in a false signal. This field component must be measured if a precision search for $d_{\mu}$ is to be undertaken. Accordingly, a novel technique was presented in Chapter \ref{chap:4} for measuring the average background radial magnetic field, $\langle B_{r}^{b} \rangle$, to a precision of $\leq1$ parts-per-million (ppm), without disturbing the storage ring vacuum or requiring the use of additional hardware. Two measurements were performed by the author in E989 Run-4: the first being preliminary measurement, which was subsequently used to adjust $\langle B_{r}^{b} \rangle$ to zero, and the second being a higher precision measurement, whereby $\langle B_{r}^{b} \rangle$ was measured to be $-0.4\pm0.6$ ppm. Based on this result, a method was designed to estimate the total average radial magnetic field, $\langle B_{r} \rangle$, in any E989 dataset, where estimates were made for the available subsets of Run-3 and Run-5, as well as the entirety of Run-1 and Run-2. This work ensures that the radial magnetic field systematic does not limit the sensitivity to $d_{\mu}$ at Fermilab, in Run-1 and beyond. The techniques developed in this chapter have subsequently been utilised, with the assistance of the author, by fellow E989 collaborators to both measure $\langle B_{r}^{b} \rangle$ in Run-5 and to continue to estimate $\langle B_{r} \rangle$ in E989 datasets as they are produced.

The measured vertical angle associated with a muon EDM is reduced (diluted) compared to the true precession plane tilt angle, and the ability to correct this effect is essential to the measurement of $d_{\mu}$. A thorough assessment of the tilt angle dilution, by use of a large-scale Monte Carlo simulation, was detailed in Chapter \ref{chap:5}. The analytical form of the dilution as a function of positron momentum, associated with both the inherent decay asymmetry and relativistic effects, was verified, and the contribution to the dilution from straw tracker acceptance was characterised. Additionally, simulation was used to make a preliminary estimation of the systematic uncertainty associated with detector global vertical position alignment, which will be expanded to incorporate the impact of global detector tilt angle. The uncertainties associated with acceptance are expected to diminish as a greater number of events are folded into the Monte Carlo dataset used to derive the acceptance correction. Additional studies on the impact of the non-typical beam conditions in Run-1 on acceptance will also be included as the analysis progresses towards an unblinding.

Finally, a blinded search for the muon EDM in E989 Run-1 was presented in Chapter \ref{chap:6}, where the primary analysis was performed in momentum intervals in order to maximise the accuracy of the dilution correction, and to minimise momentum dependent variations in the average vertical decay angle. An assessment of potential sources of systematic uncertainty in the analysis was also outlined in that chapter, including a description of the corrections necessitated by the aforementioned unique beam conditions in Run-1. A preliminary combined blind result for Run-1 was presented, with an estimated total uncertainty of $1.04\times10^{-19} \,e\cdot\text{cm}$. This uncertainty was used to estimate an expected upper limit on the muon EDM of $  | d_{\mu} | <  2.0\times10^{-19} \,e\cdot\text{cm (95\% C.L.)}$, which improves upon the upper limit from the equivalent traceback analysis performed at BNL \cite{BNLEDM}. 

The author will remain closely involved in the Run-1 muon EDM search until an agreement to remove the blind offset and move to publish the result is reached. More broadly, the work presented in this thesis lays the foundations for the greater muon EDM search across the E989 integrated dataset, which will ultimately result in either a world-leading upper limit on the muon EDM, or the discovery of new physics outright. 
